\documentclass{article}

\usepackage{ifxetex}
\usepackage{ifluatex}

\ifxetex
  \usepackage[ios,font=seguiemj.ttf]{emoji}
  \usepackage{fontspec}
  \newcommand{\dogline}{The \emoji{23E9} brown \emoji{1F43A} jumps over the lazy \emoji{1F436}.}
\else
  \ifluatex
    \usepackage[ios,font=Symbola_hint.ttf]{emoji}
    \usepackage{fontspec}
    \newcommand{\dogline}{The \emoji{23E9} brown \emoji{1F43A} jumps over the lazy \emoji{1F436}.}
  \else
    \usepackage[T1]{fontenc}
    \usepackage[utf8]{inputenc}
    \usepackage[ios]{emoji}
    \newcommand{\dogline}{The ⏩ brown 🐺 jumps over the lazy 🐶.}
  \fi
\fi

\usepackage{times}
\usepackage{graphicx}

\title{Using emoji inside \LaTeX}

\begin{document}

\maketitle

You can include any available emoji with the \verb|\emoji| command. For example, calling \verb|\emoji{1F514}| gives \emoji{1F514}. You can switch between different versions via the optional argument:
\begin{description}
\item[{\emoji[android]{1F514}} =] \verb|\emoji[android]{1F514}|
\item[{\emoji[ios]{1F514}} =] \verb|\emoji[ios]{1F514}|
\item[{\emoji[windows]{1F514}} =] \verb|\emoji[windows]{1F514}|
\item[{\emoji[bw]{1F514}} =] \verb|\emoji[bw]{1F514}|
\item[{\emoji[twitter]{1F514}} =] \verb|\emoji[twitter]{1F514}|
\end{description}
To select multi-char emoji, pass both codes joined by a dash to the command. For example, \verb|\emoji{1F1E8-1F1F3}| gives \emoji{1F1E8-1F1F3}.

For single character emoji, you can also type that character directly (encode your document in UTF-8) and the emoji will be displayed instead. For example, the line ``\dogline'' does not contain the \verb|emoji| command. Note that this is disabled when using XeTeX. 

During import of the package, you can pick the default emoji set by passing an optional argument as for the \verb|emoji| command. If you want to switch emoji sets for more than one emoji, you can use the \verb|android-emojis|, \verb|ios-emojis|, \verb|bw-emojis|, \verb|windows-emojis|, \verb|twitter-emojis| environments to temporarily change the default. For example, all following emojis are set in twitter style:
\begin{twitter-emojis}
\dogline
\end{twitter-emojis}

When using XeTeX, you can also specify a \verb|text| option to set the emoji in a given font. The font to use is passed to the package during loading. On Windows, for example, you might want to use \textit{Segeo UI Emoji} or \textit{Symbola}. That font would be selected by using: \verb|usepackage[font=seguiemj.ttf]{emoji}| or \verb|usepackage[font=Symbola_hint.ttf]{emoji}|. Emojis rendered via a font are, e.g.: 
\begin{text-emojis}
\emoji{23E9} \emoji{1F43A} \emoji{1F436}
\end{text-emojis}

\end{document}